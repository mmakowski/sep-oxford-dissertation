\documentclass[proposal]{softeng}

\usepackage[utf8]{inputenc}
\usepackage{times}
\usepackage[normalem]{ulem}
\usepackage{url}

\begin{document}

\title{A Neural Model for Software Vulnerability Prediction}
\author{Maciek Makowski}
\organisation{University of Oxford}
\college{Kellogg College}
\maketitle

\begin{abstract}
It is proposed that neural network based models for prediction of vulnerabilities in
software components are investigated. In particular, deep learnig techniques 
inspired by those used for natural language modelling and classification 
would be explored.
The challenges posed by the project include training data curation,
choice of representation and choice of the neural network architecture.
The contribution of the work would comprise assessment of applicability of
neural network models to security analysis of source code and, potentially, a novel
vulnerability model.
\end{abstract}

\section{Area of Study}

The social, political and financial impact of software security vulnerabilities, 
exemplified by recent \emph{WannaCry} malware infection that affected
UK National Health Service, calls 
for tools and techniques that aid in early discovery of such vulnerabilities. 
The spectrum of existing approaches includes techniques based on human assessment,
such as code reviews and penetration testing, as well as automated tools that perform
either dynamic or static code analysis. The focus of our work will be on the last 
group of techniques, and specifically on the application of machine learning in the
domain.

Prior work in the area involved machine learning models such as support vector
machines (SVMs)~\cite{neuhaus2007predicting,perl2015vccfinder}, logistic regression 
~\cite{hata2010fault}, naïve Bayes~\cite{hata2010fault,scandariato2014predicting} 
and random forest~\cite{scandariato2014predicting}, which require the input features 
to be hand-picked.
Recent advances in \emph{representation learning}, in particular using deep
neural networks (\emph{deep learnig}), might allow training vulnerability models 
without the oversight of a software security expert. Exploratory work by Andrej 
Karpathy~\cite{karpathy2015unreasonable} suggests that recurrent neural networks 
(RNNs) have the capacity to learn syntactic properties of source code.
This result holds promise for extending such models to classify security properties
of the code. Neural networks have been applied to security analysis of software 
design~\cite{adebiyi2013security}, but, to the best of our knowledge, not to
source code directly. This would be a new contribution of the proposed project.

The relative novelty of the deep learning models and scarcity of labelled
examples are likely to pose a difficulty in achieving state of the art accuracy of
the predictions. As a result, the primary goal of the project would be exploration
of the limitations of neural model applicability and the problem space where 
such models offer advantages over alternative code analysis approaches.

\section{Proposed Work}

\subsection{Scope}

\paragraph{Review of approaches} We will present and discuss relative merits
of existing techniques for software vulnerability detection and prediction. 
Particular attention will be paid to vulnerability 
prediction work that relies on machine learning. We will explain how
the use of representation learning can address the shortcomings of existing
methods, and the classification algorithms developed in natural language
processing (NLP) domain that might be applicable to source code analysis and
classification.

\paragraph{Data sets} Machine learning techniques rely on the availability of
training data sets. In particular, for supervised training of classifiers,
labelled data is required. The data sets curated for prior research 
(\cite{li2016vulpecker}, \cite{perl2015vccfinder}, \cite{meneely2013patch})
will be reviewed for their applicability to training neural network models. 
If required, a new data set will be constructed based on public databases of 
vulnerabilities in open source software and public source code repositories.

\paragraph{Input representation} A range of possible representations of the source
code that is taken as the input to the model is feasible: from sequences of 
characters to abstract syntax trees. We will address the trade-offs between different
representations as far as accuracy of predictions and flexibility of the models
is concerned.

\paragraph{Model evaluation} Selected algorithms will be implemented and
evaluated on the test data set. The evaluation criteria will include 
prediction accuracy, generalisation power and interpretability of the
results. Supported by this data we will reevaluate the potential area
of application for neural vulnerability models.

\subsection{Techniques}

The project will be focussed on static code analysis, with
particular attention paid to classification approaches used
in NLP that can be transferred to the domain of source code. We expect
to employ a range of machine learning techniques: recurrent neural
networks, attention, autoencoders, unsupervised pre-training and 
adversarial training. We will research visualisation approaches to
aid in interpretation of the results of the classifier.

\subsection{Expected Outcomes}

The primary outcome of the project is expected to be an evaluation
of the problem space where neural network approaches to modelling vulnerabilities
is likely to be useful. We expect that the accuracy and predictive power
of the model would not match state of the art in the general case. However,
we expect that for certain classes of bugs the neural network approach
will achieve promising results. We also anticipate that metrics other than
classification quality -- for example, ease of adaptation to new programming 
languages -- might render those models compelling.


\subsection{Risks}

While representation learning eliminates feature engineering and the associated
involvement of a domain expert or machine learning specialist, the existing 
representation learning techniques
require large volume of training data. As a reference, \cite{karpathy2015unreasonable} 
model was 
trained on aproximately 500 megabytes of C code. Obtaining this amount of labelled 
training data will most likely be infeasible, thus ruling out the most straightforward
model trainig scheme. Use of more advanced approaches, for example ones that involve
pre-training on unlabelled source code, poses a risk to the project, as those techniques
are still in the area of active research and are not very well understood. In addition
the validity of the results is threatened by lack of definitive negative examples --
lack of vulnerability report against a piece of code does not guarantee that it is
vulnerability-free.

\section{Project Plan}

The major milestones of the project and estimated elapsed time required to achieve
each of them are listed below.

\paragraph{Literature review} A survey of prior work in the field that would contribute
to the content of \emph{Review of vulnerability prediction approaches} section. 
In addition, the review will 
provide information on existing data sets, on evaluation approaches and on reference 
results for comparison with our model. It is expected to take 5 weeks.

\paragraph{Data curation} As a result of the literature review, existing data sets
to use for training and evaluation can be identified. In this phase we will obtain
those data sets, and, should they prove insufficient, curate a new one that will be
used for training the model. Approximate duration of this phase will be 8 weeks.

\paragraph{Model implementation and training} The neural network models will be 
designed, implemented and trained. The training and preliminary evaluation phase 
is expected to be time-consuming, given that the RNN programming language model in 
\cite{karpathy2015unreasonable} took a number of days
to train and multiple iterations of training will most likely be required. This phase 
is expected to take 15 weeks.


* experiments
* dissertation writing


\section{Dissertation Structure}

\begin{enumerate}\raggedright
\item Introduction
\item Vulnerability Prediction and Machine Learning
    \begin{enumerate}
    \item Context for vulnerability prediction
    \item Review of vulnerability prediction approaches
    \item Prior work on machine learning application to vulnerability analysis
    \item Natural language models and their applicability to source code
    \end{enumerate}
\item Model Implementation and Evaluation
    \begin{enumerate}
    \item Data set curation
    \item Input representation
    \item Description of implemented models
    \item Model evaluation and comparison to existing techniques
    \item Reflection on applicability
    \end{enumerate}
\item Conclusion and Future Work
\end{enumerate}

\bibliographystyle{alpha}
\bibliography{project-proposal}

\end{document}
