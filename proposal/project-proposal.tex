\documentclass[twoside,a4paper]{scrartcl}

\usepackage[utf8]{inputenc}
\usepackage[normalem]{ulem}
\usepackage{url}


\begin{document}

\title{A Neural Model for Software Vulnerability Prediction}
\subtitle{Project Proposal} 
\author{Maciek Makowski}
\maketitle

\begin{abstract}
We propose to investigate techniques for prediction of vulnerabilities in
software components that are based on neural network models that analyse
source code. In particular, deep learnig techniques inspired by those used
for natural language modelling and classification are going to be explored.
The challenges we will need to overcome include training data curation,
choice of representation and choice of the neural network architecture.
The contribution of the work would consist of a novel vulnerability model
as well as insight into source code features that correlate with security
vulnerabilities.
\end{abstract}

\section{Area of Study}

The social, political and financial impact of software security vulnerabilities, 
exemplified by recent \emph{WannaCry} malware infection that affected
United Kingdom National Health Service, calls 
for tools and techniques that aid in early discovery of such vulnerabilities. 
The spectrum of existing approaches includes techniques based on human assessment,
such as code reviews and penetration testing, as well as automated tools that perform
either dynamic or static code analysis. The focus of our work will be on the last 
group of techniques, and specifically on the application of machine learning in the
domain. While existing pattern-matching tools are effective in discovering known
vulnerability patterns in new software, the generalisation properties of machine 
learning algorithms give hope that new types of vulnerabilities can be predicted
based on source code features learned from existing examples. 

Prior work in the area involved machine learning models such as support vector
machines (SVMs) \cite{neuhaus2007predicting,perl2015vccfinder}, naïve Bayes and 
random forest
\cite{scandariato2014predicting}, which require the input features to be hand-picked.
Recent advances in \emph{representation learning}, in particular using deep
neural networks (\emph{deep learnig}), might allow training vulnerability models 
without the oversight of a software security expert. Exploratory work by Andrej 
Karpathy \cite{karpathy2015unreasonable} suggests that recurrent neural networks 
(RNNs) have the capacity to learn syntactic properties of source code.
This result holds promise for extending such models to classify security properties
of the code. Neural networks have been applied to security analysis of software 
design \cite{adebiyi2013security}, but, to the best of our knowledge, not to
source code directly. This would be a new contribution of this project.

The relative novelty of the deep learning models and scarcity of labelled
examples are likely to pose a difficulty in achieving state-of-the-art accuracy of
the predictions. Nonetheless, the work will explore the limitations of
neural model applicability and the problem space where such models offer advantages
over alternative code analysis approaches.

\section{Proposed Work}

\subsection{Scope}

\paragraph{Data sets} Data sets curated for prior research (\cite{li2016vulpecker}, 
\cite{perl2015vccfinder}, \cite{meneely2013patch})
will be reviewed. If required, a comprehensive data set will be constructed
such set based on public databases of vulnerabilities in open-source software. 

\paragraph{Input representation} A range of possible representations of the source
code that is taken as the input to the model is feasible: from sequences of 
characters to abstract syntax trees. We will address the trade-offs between different
representations.

* choice of the model (RNN/attention) -- focus on both quality and interpretability
* implementation of the model
* interpretation of results

\subsection{Techniques}

\paragraph{Data set augmentation} Since it is expected that labelled examples will 
be scarce, we will investigate techniques for data set augmentation, including 
vulnerability-invariant source code transformations. 

\paragraph{Pre-training} In the likely event that insufficient labelled training data is
present, we will apply unsupervised pre-training techniques to improve the model accuracy.

\paragraph{Recurrent neural network}

* interpretability: attention visualisation of the relevant features

\subsection{Expected Outcomes}

* quality: probably not state of the art
* insight into source code features that are correlated with sec vulnerabilities

\subsection{Risks}

While representation learning eliminates feature engineering and the associated
involvement of a domain expert or machine learning specialist, the existing 
representation learning techniques
require large volume of training data. As a reference, \cite{karpathy2015unreasonable} 
model was 
trained on aproximately 500 megabytes of C code. Obtaining this amount of labelled 
training data will most likely be infeasible, thus ruling out the most straightforward
model trainig scheme. Use of more advanced approaches, for example ones that involve
pre-training on unlabelled source code, poses a risk to the project, as those techniques
are still in the area of active research and are not very well understood. In addition
the validity of the results is threatened by lack of definitive negative examples --
lack of vulnerability report against a piece of code does not guarantee that it is
vulnerability-free.

\section{Project Plan}

The major milestones of the project and estimated effort involved are described below.

\paragraph{Literature review} A survey of prior work in the field that would contribute
to the content of \emph{Review of Prior Work} section. In addition, the review will 
provide information on existing data sets, on evaluation approaches and on reference 
results for comparison with our model. It is expected to take 5 weeks.

\paragraph{Data curation} As a result of the literature review, existing data sets
to use for training and evaluation can be identified. In this phase we will obtain
those data sets, and, should they prove insufficient, curate a new one that will be
used for training the model. Approximate duration of this phase will be 8 weeks.

\paragraph{Model implementation and training} The neural network model will be 
designed, implemented and trained. The training and preliminary evaluation phase 
is expected to be time-consuming, given that the RNN programming language model in 
\cite{karpathy2015unreasonable} took a number of days
to train and multiple iterations of training will most likely be required. This phase 
is expected to take 15 weeks.


* experiments
* dissertation writing


\section{Dissertation Structure}

\begin{enumerate}
\item Introduction
\item Review of Prior Work
\item Neural Models and Deep Learning
    \begin{enumerate}
    \item Natural Language
    \item Programming Language
    \end{enumerate}
\item Input Representation
    \begin{enumerate}
    \item Bag-of-Words
    \item Character Sequence
    \item Token Sequence
    \item Abstract Syntax Tree
    \end{enumerate}
\item Data Sets
\item Model Selection
\item Evaluation
    \begin{enumerate}
    \item Accuracy
    \item Generalisation
    \item Interpretation of Results
    \end{enumerate}
\item Conclusion and Future Work
\end{enumerate}

\bibliographystyle{alpha}
\bibliography{project-proposal}

\end{document}
